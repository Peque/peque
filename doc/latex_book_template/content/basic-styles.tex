
\cquote{A quien no sabe a qué puerto se dirige, ¡mal viento lo lleve!}{Maxi Peque}

\section{Paragraphs, lists}

\subsection{Paragraphs}

Australia, Australia sí que es la hostia. ¿Tú sabes cuantos kilómetros cuadrados tiene Australia? Siete millones seiscientos y pico mil. Diez veces esto. ¿Y habitantes? Ni la mitad que aquí. Así que calcula. Calcula a lo que tocan por cabeza. Aquí no salimos a una mierda.

Porque te dan tu parte. Cuando te jubilas. Por una ley que hay. Dividen. Tantos Kilómetros de país los que sean, por tantas personas, tanto. No sé, ponle. Dos kilómetros cuadrados, tres, lo que toque. Y te lo dan. A cada uno su trozo, ¿Te imaginas? Toma, pum, lo tuyo. Para ti para siempre. Y tú haces ahí lo que te sale de los huevos.

\subsection{Lists}

The first formal definition of free software was published by FSF in February 1986. That definition, written by Richard Stallman, is still maintained today and states that software is free software if people who receive a copy of the software have the following four freedoms.

\begin{itemize}
  \item The freedom to run the program for any purpose.
  \item The freedom to study how the program works, and change it to make it do what you wish.
  \item The freedom to redistribute copies so you can help your neighbor.
  \item The freedom to improve the program, and release your improvements (and modified versions in general) to the public, so that the whole community benefits.
\end{itemize}

